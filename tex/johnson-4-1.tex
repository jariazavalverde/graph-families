% Johnson graph n=4 k=1 J(4,1)
% Author: José Antonio Riaza Valverde
% <https://github.com/jariazavalverde/graphs>
\documentclass{article}
\usepackage{tikz}
\usetikzlibrary{automata}

\begin{document}
\begin{tikzpicture}[node distance=2.5cm, line width=0.4mm,
	red/.style={circle,draw=red!100,fill=red!100,inner sep=0pt, minimum size=0.2cm},
	gray/.style={circle,draw=black!30,fill=black!30,inner sep=0pt, minimum size=0.2cm}]
	\node[state] (a) {
		\begin{tikzpicture}[node distance=0.4cm, radius=1cm]
			\node[red] (a1) {};
			\node[gray] (a2) [left of=a1] {};
			\node[gray] (a3) [below of=a2] {};
			\node[gray] (a4) [right of=a3] {};
    	\end{tikzpicture}	
	};
	\node[state] (b) [left of=a] {
		\begin{tikzpicture}[node distance=0.4cm, radius=1cm]
			\node[gray] (a1) {};
			\node[red] (a2) [left of=a1] {};
			\node[gray] (a3) [below of=a2] {};
			\node[gray] (a4) [right of=a3] {};
    	\end{tikzpicture}	
	};
	\node[state] (c) [below of=b] {
		\begin{tikzpicture}[node distance=0.4cm, radius=1cm]
			\node[gray] (a1) {};
			\node[gray] (a2) [left of=a1] {};
			\node[red] (a3) [below of=a2] {};
			\node[gray] (a4) [right of=a3] {};
    	\end{tikzpicture}	
	};
	\node[state] (d) [right of=c] {
		\begin{tikzpicture}[node distance=0.4cm, radius=1cm]
			\node[gray] (a1) {};
			\node[gray] (a2) [left of=a1] {};
			\node[gray] (a3) [below of=a2] {};
			\node[red] (a4) [right of=a3] {};
    	\end{tikzpicture}	
	};
	\path (a) edge node {} (b)
	      (a) edge node {} (c)
	      (a) edge node {} (d)
	      (b) edge node {} (c)
	      (b) edge node {} (d)
	      (c) edge node {} (d);
\end{tikzpicture}
\end{document}